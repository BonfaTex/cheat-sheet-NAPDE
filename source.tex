%!TEX root = foglio.tex

% =================================================

\section{Diffusion/Elliptic Problems}

% =================================================

\textbf{Weak Formulation.} Given $V$, $F:V\to\RR$ functional and $a:V\times V\to\RR$ bilinear form,
\begin{equation*}
\boxed{\text{find } u\in V\ :\ a(u,v)=F(v)\quad\forall v\in V}\qquad\quad
\end{equation*}

\smallskip

\textbf{Lax-Milgram Lemma.} Assume that:
\begin{enumerate}
\item $V$ Hilbert space with $\norm{\cdot}_V$ and $\sca{\cdot,\cdot}$
\item $F\in V'$, i.e. $\abs{F(v)}\leq \norm{F}_{V'}\norm{v}_V$
\item $a$ cont., i.e. $\abs{a(u,v)}\leq M \norm{u}_V\norm{v}_V$
\item $a$ coercive, i.e. $a(v,v)\geq \alpha \norm{v}_V^2$
\end{enumerate}
Then: $\exists\,!\ u$ sol. di WF, and $\norm{u}_V\leq \nicefrac{\norm{F}_{V'}}{\alpha}$

\medskip

\textbf{Galerkin Approximation.} If you can build $V_h\subset V$ s.t. $\text{dim }V_h=N_h<\infty$ ($\Rightarrow$ $V_h$ closed subspace), then WF becomes G:
\begin{equation*}
\boxed{\text{find } u_h\in V_h\ :\ a(u_h,v_h)=F(v_h)\quad\forall v_h\in V_h}\qquad
\end{equation*}

\begin{itemize}
\item \emph{well-posedness} follows from LM
\item \emph{stability} is the continuous dependence from data in LM
\item \emph{consistency} $\equiv$ \textbf{Galerkin Orthogonality}:
\begin{equation*}
a(u-u_h,v_h)=0\qquad\forall v_h\in V_h
\end{equation*}
\item if we assume \textbf{space saturation}
\begin{equation*}
\inf_{v_h\in V_h}\norm{v-v_h}_V=0\quad\forall v\in V
\end{equation*}
then \emph{convergence} $\equiv$ \textbf{Céa Lemma}:
\begin{equation*}
\norm{u-u_h}_V\leq \frac{M}{\alpha}\,\inf_{v_h\in V_h}\norm{u-v_h}_V
\end{equation*}
(Céa + space saturation $\equiv$ convergence)
\end{itemize}

Last but not least: Problem G is equivalent to the following linear system of equations:
\begin{equation*}
\boxed{\text{find } \uv\in\RR^{N_h}\ :\ A\uv=\Fv}\qquad\qquad
\end{equation*}
where $A\in\RR^{N_h\times N_h},\ \Fv\in\RR^{N_h}$.

\begin{proof}
$V_h=\text{span}\left\{ \phi_1,\dots,\phi_{N_h} \right\}$ so
\begin{equation*}
u_h(\xv)=\sum_{j=1}^{N_h} U_j \phi_j(\xv),\quad U_j\in\RR\ \forall j
\end{equation*}
thus G becomes: Find $\left\{ U_j \right\}_{j=1}^{N_h}$ s.t.
\begin{gather*}
a\left( \sum_j U_j\phi_j,\phi_i \right)=F( \phi_i )\quad\forall\, i=1:N_h \\
\sum_j U_j\, a(\phi_j,\phi_i) = F(\phi_i)\quad\forall\, i=1:N_h \\
\sum_j A_{ij}\,U_j=F_i\quad\forall\, i=1:N_h \\
A\Uv=\Fv
\end{gather*}
\end{proof}

Moral of the story: $V_h$ must be chosen to ensure the saturation assumption and the computation of the integrals $A_{ij}=a(\phi_j,\phi_i)$ and $F_i=F(\phi_i)$.

\medskip

\textbf{The Finite Element Method.}

\bigskip

\bigskip

\bigskip

\bigskip

\bigskip

\bigskip

\bigskip

\bigskip

\bigskip

\bigskip

\bigskip

\bigskip

\bigskip

\bigskip

\bigskip

\bigskip

\bigskip

\bigskip

\bigskip

\bigskip

\bigskip

\bigskip

\bigskip

\bigskip

\bigskip

\bigskip

\bigskip

\bigskip

\bigskip

\bigskip

\bigskip

\bigskip

\bigskip

\bigskip

\bigskip

\bigskip

\bigskip

\bigskip

\bigskip

\bigskip

\bigskip

\bigskip

\bigskip

\bigskip

\bigskip

\bigskip

\bigskip

\bigskip

\bigskip

\bigskip

\bigskip

\bigskip

\bigskip

\bigskip

\bigskip

\bigskip

\bigskip

\bigskip

\bigskip

\bigskip

\bigskip

\bigskip

\bigskip

\bigskip

\bigskip

\bigskip

\bigskip

\bigskip

\bigskip

\bigskip

\bigskip

\bigskip

\bigskip

\bigskip

\bigskip

\bigskip

\bigskip

\bigskip

\bigskip

\bigskip

\bigskip

\bigskip

\bigskip

\bigskip

\bigskip

\bigskip

\bigskip

\bigskip

\bigskip

\bigskip

\bigskip

\bigskip

\bigskip

\bigskip

\bigskip

\bigskip

\bigskip

\bigskip

\bigskip

\bigskip

\bigskip

a






